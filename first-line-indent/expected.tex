% Options for packages loaded elsewhere
\PassOptionsToPackage{unicode}{hyperref}
\PassOptionsToPackage{hyphens}{url}
%
\documentclass[
]{article}
\usepackage{amsmath,amssymb}
\usepackage{lmodern}
\usepackage{iftex}
\ifPDFTeX
  \usepackage[T1]{fontenc}
  \usepackage[utf8]{inputenc}
  \usepackage{textcomp} % provide euro and other symbols
\else % if luatex or xetex
  \usepackage{unicode-math}
  \defaultfontfeatures{Scale=MatchLowercase}
  \defaultfontfeatures[\rmfamily]{Ligatures=TeX,Scale=1}
\fi
% Use upquote if available, for straight quotes in verbatim environments
\IfFileExists{upquote.sty}{\usepackage{upquote}}{}
\IfFileExists{microtype.sty}{% use microtype if available
  \usepackage[]{microtype}
  \UseMicrotypeSet[protrusion]{basicmath} % disable protrusion for tt fonts
}{}
\usepackage{xcolor}
\usepackage{color}
\usepackage{fancyvrb}
\newcommand{\VerbBar}{|}
\newcommand{\VERB}{\Verb[commandchars=\\\{\}]}
\DefineVerbatimEnvironment{Highlighting}{Verbatim}{commandchars=\\\{\}}
% Add ',fontsize=\small' for more characters per line
\newenvironment{Shaded}{}{}
\newcommand{\AlertTok}[1]{\textcolor[rgb]{1.00,0.00,0.00}{\textbf{#1}}}
\newcommand{\AnnotationTok}[1]{\textcolor[rgb]{0.38,0.63,0.69}{\textbf{\textit{#1}}}}
\newcommand{\AttributeTok}[1]{\textcolor[rgb]{0.49,0.56,0.16}{#1}}
\newcommand{\BaseNTok}[1]{\textcolor[rgb]{0.25,0.63,0.44}{#1}}
\newcommand{\BuiltInTok}[1]{\textcolor[rgb]{0.00,0.50,0.00}{#1}}
\newcommand{\CharTok}[1]{\textcolor[rgb]{0.25,0.44,0.63}{#1}}
\newcommand{\CommentTok}[1]{\textcolor[rgb]{0.38,0.63,0.69}{\textit{#1}}}
\newcommand{\CommentVarTok}[1]{\textcolor[rgb]{0.38,0.63,0.69}{\textbf{\textit{#1}}}}
\newcommand{\ConstantTok}[1]{\textcolor[rgb]{0.53,0.00,0.00}{#1}}
\newcommand{\ControlFlowTok}[1]{\textcolor[rgb]{0.00,0.44,0.13}{\textbf{#1}}}
\newcommand{\DataTypeTok}[1]{\textcolor[rgb]{0.56,0.13,0.00}{#1}}
\newcommand{\DecValTok}[1]{\textcolor[rgb]{0.25,0.63,0.44}{#1}}
\newcommand{\DocumentationTok}[1]{\textcolor[rgb]{0.73,0.13,0.13}{\textit{#1}}}
\newcommand{\ErrorTok}[1]{\textcolor[rgb]{1.00,0.00,0.00}{\textbf{#1}}}
\newcommand{\ExtensionTok}[1]{#1}
\newcommand{\FloatTok}[1]{\textcolor[rgb]{0.25,0.63,0.44}{#1}}
\newcommand{\FunctionTok}[1]{\textcolor[rgb]{0.02,0.16,0.49}{#1}}
\newcommand{\ImportTok}[1]{\textcolor[rgb]{0.00,0.50,0.00}{\textbf{#1}}}
\newcommand{\InformationTok}[1]{\textcolor[rgb]{0.38,0.63,0.69}{\textbf{\textit{#1}}}}
\newcommand{\KeywordTok}[1]{\textcolor[rgb]{0.00,0.44,0.13}{\textbf{#1}}}
\newcommand{\NormalTok}[1]{#1}
\newcommand{\OperatorTok}[1]{\textcolor[rgb]{0.40,0.40,0.40}{#1}}
\newcommand{\OtherTok}[1]{\textcolor[rgb]{0.00,0.44,0.13}{#1}}
\newcommand{\PreprocessorTok}[1]{\textcolor[rgb]{0.74,0.48,0.00}{#1}}
\newcommand{\RegionMarkerTok}[1]{#1}
\newcommand{\SpecialCharTok}[1]{\textcolor[rgb]{0.25,0.44,0.63}{#1}}
\newcommand{\SpecialStringTok}[1]{\textcolor[rgb]{0.73,0.40,0.53}{#1}}
\newcommand{\StringTok}[1]{\textcolor[rgb]{0.25,0.44,0.63}{#1}}
\newcommand{\VariableTok}[1]{\textcolor[rgb]{0.10,0.09,0.49}{#1}}
\newcommand{\VerbatimStringTok}[1]{\textcolor[rgb]{0.25,0.44,0.63}{#1}}
\newcommand{\WarningTok}[1]{\textcolor[rgb]{0.38,0.63,0.69}{\textbf{\textit{#1}}}}
\usepackage{longtable,booktabs,array}
\usepackage{calc} % for calculating minipage widths
% Correct order of tables after \paragraph or \subparagraph
\usepackage{etoolbox}
\makeatletter
\patchcmd\longtable{\par}{\if@noskipsec\mbox{}\fi\par}{}{}
\makeatother
% Allow footnotes in longtable head/foot
\IfFileExists{footnotehyper.sty}{\usepackage{footnotehyper}}{\usepackage{footnote}}
\makesavenoteenv{longtable}
\setlength{\emergencystretch}{3em} % prevent overfull lines
\providecommand{\tightlist}{%
  \setlength{\itemsep}{0pt}\setlength{\parskip}{0pt}}
\setcounter{secnumdepth}{-\maxdimen} % remove section numbering
\setlength{\parindent}{2em}
\ifLuaTeX
  \usepackage{selnolig}  % disable illegal ligatures
\fi
\IfFileExists{bookmark.sty}{\usepackage{bookmark}}{\usepackage{hyperref}}
\IfFileExists{xurl.sty}{\usepackage{xurl}}{} % add URL line breaks if available
\urlstyle{same} % disable monospaced font for URLs
\hypersetup{
  pdftitle={Sample first line indent},
  hidelinks,
  pdfcreator={LaTeX via pandoc}}

\title{Sample first line indent}
\author{}
\date{}

\begin{document}
\maketitle

\noindent This document tests the first-line indent mode. In
English-style typography, a first-line indent is only applied when
needed to separate a paragraph from the preceding paragraph. Hence there
is no first-line indent below section or chapter headings (as opposed to
French-style typography). By the same logic, the first paragraph below
an article title and the first paragraph of a document without title
should not be indented. However, LaTeX classes indent them. We follow
this practice here - so this paragraph should be indented. This
paragraph should start with a first-line indent. But after this quote:

\begin{quote}
Lorem ipsum dolor sit amet, consectetur adipiscing elit.
\end{quote}

\noindent the paragraph continues, so there should not be a first-line
indent.

The quote below ends a paragraph:

\begin{quote}
Lorem ipsum dolor sit amet, consectetur adipiscing elit.
\end{quote}

\indent This paragraph below, then, is properly a new paragraph and
starts with a first-line indent which we have to manually specify with
\texttt{\textbackslash{}indent}.

\hypertarget{further-tests}{%
\section{Further tests}\label{further-tests}}

After a heading (in English typographic style) the paragraph does not
have a first-line indent.

In the couple couple of paragraphs that follow the quotes below, we have
manually specified \texttt{\textbackslash{}noindent} and
\texttt{\textbackslash{}indent} respectively. This is to check that the
filter doesn't add its own commands to those.

\begin{quote}
Lorem ipsum dolor sit amet, consectetur adipiscing elit.
\end{quote}

\noindent Manually specified no first line indent.

\indent Manually specified first line ident.

We can also check that indent is removed after lists:

\begin{itemize}
\tightlist
\item
  A bullet
\item
  list
\end{itemize}

\noindent And after code blocks:

\begin{Shaded}
\begin{Highlighting}[]
\KeywordTok{local}\NormalTok{ variable }\OperatorTok{=} \StringTok{"value"}
\end{Highlighting}
\end{Shaded}

\noindent Or horizontal rules.

\begin{center}\rule{0.5\linewidth}{0.5pt}\end{center}

\noindent We can check that default behaviour is overridden for elements
of certain classes, adding indent:

\begin{Shaded}
\begin{Highlighting}[]
\NormalTok{This code block should be followed by an indented paragraph.}
\end{Highlighting}
\end{Shaded}

Or removing it:

This Div should be followed by a paragraph without indent.

\noindent as desired.

In this document we added a few custom filter options. The size of
first-line indents is 2em instead of the standard 1em. We also added an
option to remove indent after tables:

\begin{longtable}[]{@{}rlcl@{}}
\caption{Demonstration of simple table syntax.}\tabularnewline
\toprule()
Right & Left & Center & Default \\
\midrule()
\endfirsthead
\toprule()
Right & Left & Center & Default \\
\midrule()
\endhead
12 & 12 & 12 & 12 \\
123 & 123 & 123 & 123 \\
1 & 1 & 1 & 1 \\
\bottomrule()
\end{longtable}

\noindent So this paragraph's first line is not indented. And we
included custom options \emph{not} to remove ident after ordered lists
and definition lists:

\begin{description}
\tightlist
\item[Definition]
This is a definition block.
\end{description}

This paragraph is indented.

\begin{enumerate}
\def\labelenumi{\arabic{enumi}.}
\tightlist
\item
  An ordered
\item
  list
\end{enumerate}

This paragraph is indented.

\end{document}
