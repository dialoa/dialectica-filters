% Options for packages loaded elsewhere
\PassOptionsToPackage{unicode}{hyperref}
\PassOptionsToPackage{hyphens}{url}
%
\documentclass[
]{article}
\usepackage{amsmath,amssymb}
\usepackage{lmodern}
\usepackage{iftex}
\ifPDFTeX
  \usepackage[T1]{fontenc}
  \usepackage[utf8]{inputenc}
  \usepackage{textcomp} % provide euro and other symbols
\else % if luatex or xetex
  \usepackage{unicode-math}
  \defaultfontfeatures{Scale=MatchLowercase}
  \defaultfontfeatures[\rmfamily]{Ligatures=TeX,Scale=1}
\fi
% Use upquote if available, for straight quotes in verbatim environments
\IfFileExists{upquote.sty}{\usepackage{upquote}}{}
\IfFileExists{microtype.sty}{% use microtype if available
  \usepackage[]{microtype}
  \UseMicrotypeSet[protrusion]{basicmath} % disable protrusion for tt fonts
}{}
\makeatletter
\@ifundefined{KOMAClassName}{% if non-KOMA class
  \IfFileExists{parskip.sty}{%
    \usepackage{parskip}
  }{% else
    \setlength{\parindent}{0pt}
    \setlength{\parskip}{6pt plus 2pt minus 1pt}}
}{% if KOMA class
  \KOMAoptions{parskip=half}}
\makeatother
\usepackage{xcolor}
\IfFileExists{xurl.sty}{\usepackage{xurl}}{} % add URL line breaks if available
\IfFileExists{bookmark.sty}{\usepackage{bookmark}}{\usepackage{hyperref}}
\hypersetup{
  pdftitle={Consistency, Obligations, and Accuracy-Dominance Vindications},
  pdfkeywords={Epistemic
Rationality, Consistency, Veritism, Dominance, Epistemic
Normativity, Reasons},
  hidelinks,
  pdfcreator={LaTeX via pandoc}}
\urlstyle{same} % disable monospaced font for URLs
\usepackage{longtable,booktabs,array}
\usepackage{calc} % for calculating minipage widths
% Correct order of tables after \paragraph or \subparagraph
\usepackage{etoolbox}
\makeatletter
\patchcmd\longtable{\par}{\if@noskipsec\mbox{}\fi\par}{}{}
\makeatother
% Allow footnotes in longtable head/foot
\IfFileExists{footnotehyper.sty}{\usepackage{footnotehyper}}{\usepackage{footnote}}
\makesavenoteenv{longtable}
\setlength{\emergencystretch}{3em} % prevent overfull lines
\providecommand{\tightlist}{%
  \setlength{\itemsep}{0pt}\setlength{\parskip}{0pt}}
\setcounter{secnumdepth}{-\maxdimen} % remove section numbering
\makeatletter
\@ifpackageloaded{subfig}{}{\usepackage{subfig}}
\@ifpackageloaded{caption}{}{\usepackage{caption}}
\captionsetup[subfloat]{margin=0.5em}
\AtBeginDocument{%
\renewcommand*\figurename{Figure}
\renewcommand*\tablename{Table}
}
\AtBeginDocument{%
\renewcommand*\listfigurename{List of Figures}
\renewcommand*\listtablename{List of Tables}
}
\newcounter{pandoccrossref@subfigures@footnote@counter}
\newenvironment{pandoccrossrefsubfigures}{%
\setcounter{pandoccrossref@subfigures@footnote@counter}{0}
\begin{figure}\centering%
\gdef\global@pandoccrossref@subfigures@footnotes{}%
\DeclareRobustCommand{\footnote}[1]{\footnotemark%
\stepcounter{pandoccrossref@subfigures@footnote@counter}%
\ifx\global@pandoccrossref@subfigures@footnotes\empty%
\gdef\global@pandoccrossref@subfigures@footnotes{{##1}}%
\else%
\g@addto@macro\global@pandoccrossref@subfigures@footnotes{, {##1}}%
\fi}}%
{\end{figure}%
\addtocounter{footnote}{-\value{pandoccrossref@subfigures@footnote@counter}}
\@for\f:=\global@pandoccrossref@subfigures@footnotes\do{\stepcounter{footnote}\footnotetext{\f}}%
\gdef\global@pandoccrossref@subfigures@footnotes{}}
\@ifpackageloaded{float}{}{\usepackage{float}}
\floatstyle{ruled}
\@ifundefined{c@chapter}{\newfloat{codelisting}{h}{lop}}{\newfloat{codelisting}{h}{lop}[chapter]}
\floatname{codelisting}{Listing}
\newcommand*\listoflistings{\listof{codelisting}{List of Listings}}
\makeatother
\ifLuaTeX
  \usepackage{selnolig}  % disable illegal ligatures
\fi

\title{Consistency, Obligations, and Accuracy-Dominance
Vindications\thanks{This research was supported by the Social Sciences
and Humanities Research Council (grant \#756-2019-0133). Thanks to
Samuel Dishaw, Branden Fitelson, Daniel Laurier and Justin Snedegar for
helpful comments.}}
\author{true}
\date{}

\begin{document}
\maketitle
\begin{abstract}
Vindicating the claim that agents ought to be consistent has proved to
be a difficult task. Recently, some have argued that we can use
accuracy-dominance arguments to vindicate the normativity of such
requirements. But what do these arguments prove, exactly? In this paper,
I argue that we can make a distinction between two theses on the
normativity of consistency: the view that one ought to be consistent and
the view that one ought to avoid being inconsistent. I argue that
accuracy-dominance arguments for consistency support the latter view,
but not necessarily the former. I also argue that the distinction
between these two theses matters in the debate on the normativity of
epistemic rationality. Specifically, the distinction suggests that there
are interesting alternatives to vindicating the strong claim that one
ought to be consistent.
\end{abstract}

The normativity of the following formal coherence requirements is
contentious:

\textbf{Belief Consistency}. If A believes that P, it is false that A
believes that \(\neg\)P.\footnote{This requirement is sometimes called
  ``Pairwise Consistency'', as in {[}@easwaran:2016b{]}.}

\textbf{Credal Consistency}. If A has a credence of X in P, then A has a
credence of (1-X) in \(\neg\)P.

\noindent Do we fall under an obligation to satisfy these
requirements?\footnote{See {[}@way\_j:2010{]} for an overview of this
  debate. See {[}@fitelson:2016{]} on epistemic teleology and coherence
  requirements. See {[}@debona\_g-staffel:2018{]} on accuracy and
  approximation of Bayesian requirements of probabilistic coherence. See
  also Pettigrew {[}@pettigrew:2013a; @pettigrew:2016{]}.} Many
philosophers like John Broome {[}@broome:2013, ch.~13{]} are convinced
that the above requirements are normative, but cannot find a
satisfactory argument in favour of such a conclusion. Other philosophers
are less optimistic. For instance, Niko Kolodny {[}@kolodny:2005;
@kolodny:2007; @kolodny:2007a, 230--231{]} has argued that there is no
reason to be consistent. According to him, what matters from an
epistemic point of view is acquiring true beliefs (or acquiring beliefs
that are likely to be true on the evidence) and avoiding false beliefs
(or avoiding beliefs that are likely to be false on the evidence).
However, a perfectly consistent system of beliefs (or credences) can be
entirely false, inaccurate or improbable on the evidence. So,
consistency requirements are not normative, in the sense that one does
not necessarily have a reason to be consistent.

Recently, a new strategy has emerged to vindicate the normativity of
Consistency. This strategy relies on accuracy-dominance principles,
which roughly say that if state \(Y\) is better than state \(X\) at
every possible world, one ought to avoid state \(X\). However, there is
a weak and a strong interpretation of what is entailed by the
accuracy-dominance arguments. According to the strong interpretation,
accuracy-dominance arguments entail that one ought to be consistent.
Joyce, for instance, argues that:

\begin{quote}
It is thus established that degrees of belief that violate the laws of
probability are invariably less accurate than they could be. Given that
an epistemically rational agent will always strive to hold partial
beliefs that are as accurate as possible, this vindicates the
fundamental dogma of probabilism {[}according to which degrees of belief
must make conformity to the axioms of probability{]}.
{[}@joyce\_jm:1998, 600{]}
\end{quote}

According to the weak interpretation, accuracy-dominance arguments
merely entail that ought not to be inconsistent. Easwaran, for instance,
says that ``we can use dominance to \emph{eliminate}'' the inconsistent
doxastic options {[}@easwaran:2016b, 826, emphasis added{]}. In other
words, dominance is here used to argue against inconsistency. Thus, we
can make the following distinction between two views:

\textbf{Normativity+}. Given the accuracy-dominance arguments, A ought
to be consistent.

\textbf{Normativity-}. Given the accuracy-dominance arguments, A ought
not to be inconsistent.

\noindent This paper argues that, while accuracy-dominance arguments can
vindicate Normativity-, they do not necessarily vindicate Normativity+.
Specifically, accuracy-dominance arguments vindicate Normativity+ when
supplemented with a contentious hypothesis concerning the relationship
between reasons for and reasons against. Hence, accuracy-dominance
arguments do not vindicate Normativity+ \emph{on their own}.

In Sec.~\ref{sec:PREFIXthe-why-be-consistent-challenges}, I clarify the
debate on the normativity of Consistency. In Sections
\ref{sec:PREFIXaccuracy-dominance-and-consistency} and
\ref{sec:PREFIXtruth-conduciveness-reasons-for-and-reasons-against}, I
present two important arguments in the debate surrounding the
normativity of Consistency: accuracy-dominance arguments and Kolodny's
objection from truth-conduciveness. Both arguments are veritistic: They
assume that only true beliefs bear final epistemic value, and only false
beliefs bear final epistemic disvalue. I argue that, under the
assumption that veritism is true, the only way to make sense of both
arguments is to make a distinction between Normativity+ and Normativity-
(i.e.~to deny that both views are coextensive). Then, I argue that
accuracy-dominance arguments fail to vindicate Normativity+.

This is not necessarily bad news. In conclusion, I explain why this
might be an occasion to adjust our expectations in the debate on the
normativity of formal coherence requirements. Many people think that
there is something bad or suboptimal with inconsistent combinations of
attitudes. The mistake might have been to try to explain this assumption
in terms of \emph{an obligation to be consistent}. Being in a position
to vindicate Normativity- while remaining neutral on Normativity+ could
be advantageous in the debate on the normativity of formal coherence
requirements.

\hypertarget{PREFIXthe-why-be-consistent-challenges}{%
\section{The ``Why-Be-Consistent?''
Challenges}\label{PREFIXthe-why-be-consistent-challenges}}

There are many putative explanations of why one ought to have
\emph{some} consistent combinations of beliefs. They stem from the
normative authority of truth, knowledge or reasons, as in the following:

\textbf{Truth Vindication}. One ought to believe \(P\) if and only if P.
Truth is consistent (or: Inconsistent propositions cannot be true
simultaneously). So, one ought to have some consistent combinations of
beliefs (e.g.~the true ones).

\textbf{Knowledge Vindication}. One is epistemically permitted to
believe P if and only if one is in a position to know that P. Knowledge
is consistent (or: Propositions that one is in a position to know cannot
be inconsistent with each other). So, one is only epistemically
permitted to believe consistent combinations of beliefs.

\textbf{Reasons Vindication}. One is epistemically permitted to believe
\(P\) if and only if one has sufficient epistemic reason to believe P.
One never has sufficient epistemic reason to believe \(P\) and
sufficient epistemic reason to disbelieve \(P\) simultaneously. So, one
is only epistemically permitted to believe consistent combinations of
beliefs.\footnote{{[}@kolodny:2007{]} endorses this view. See
  {[}@daoust\_mk:2020{]} for discussion.}

\noindent Philosophers like {[}@broome:2013{]} and others are worried
that the above putative vindications do not fully vindicate the
normativity of Consistency. Some consistent combinations of beliefs may
include some false, unjustified or unreasonable beliefs. Even if
consistent agents sometimes believe propositions that are false,
unjustified or unreasonable, it seems that they satisfy a distinct
obligation to have consistent beliefs (e.g.~an obligation that does not
boil down to truth, knowledge or reasons). In other words, perhaps the
agent is unjustified, mistaken or unreasonable, but one could still say:
\emph{At least he or she is consistent}. Here, the putative obligation
to be consistent will not come from truth, knowledge or
reasons.\footnote{In fact, {[}@broome:2013ch.~11{]} is interested in the
  stronger claim that rationality is a \emph{source} of normativity. So,
  he is not interested in offering a derivative vindication of
  consistency requirements, that is, a vindication of these requirements
  on other grounds (like truth, knowledge, or reasons). By contrast,
  dominance principles are often tied to rationality {[}see
  e.g.@joyce\_jm:1998{]}.}

So, according to some philosophers, the above vindications are somehow
incomplete. Perhaps we can easily argue that agents ought to have
\emph{some} consistent combinations of beliefs, but finding a
vindication of Consistency that covers all the possible consistent
combinations of beliefs has proved to be a difficult task.

It should also be noted that the normativity of Consistency is part of a
broader debate on the normativity of \emph{structural rationality}.
Structural rationality allegedly requires of agents not to be incoherent
--- for example, not to be akratic, not to have intransitive
preferences, and so forth {[}@worsnip:2018b; @worsnip:2018{]}. So, in
addition to Consistency, there are other putative structural
requirements of rationality, like:

\textbf{Inter-Level Coherence.} Rationality requires that, if A believes
that he or she has sufficient epistemic reason to believe P, then A
believes that P.\footnote{{[}@coates\_a:2012{]} and
  {[}@lasonenaarnio:2020{]} have argued that responding correctly to
  one's evidence sometimes entail believing ``P, but I am irrational to
  believe P'', which is an incoherent combination of attitudes. They
  conclude that such incoherence is not necessarily irrational. See
  {[}@greco\_d:2014{]}, {[}@horowitz\_s:2014a{]},
  {[}@kiesewetter\_b:2016a{]}, {[}@littlejohn:2018a{]},
  {[}@titelbaum:2015{]} and {[}@worsnip:2018b{]} for various responses
  to this view.}

\textbf{Instrumental Principle}. Rationality requires that, if A intends
to \(\phi\), and A believes that \(\psi\)-ing is a necessary means to
\(\phi\)-ing, then A intends to \(\psi\).\footnote{See, among others,
  {[}@broome:2013sec.~9.4{]}, {[}@kiesewetter\_b:2017ch.~10{]} and
  {[}@way\_j:2013a{]} on the Instrumental Principle.}

\noindent Broome and others have tried to find compelling arguments for
the claim that \emph{structural rationality} has normative authority.
However, structural rationality is neutral on whether one's beliefs
should be true, reasonable or amount to knowledge. Some entirely false
and unreasonable belief systems can satisfy the requirements of
structural rationality. So, at least given the agenda of these
philosophers, a good vindication of the normativity of Consistency
should cover the cases in which one's beliefs are false or unreasonable.

An interesting feature of accuracy-dominance arguments is that they
remain neutral on whether one's beliefs should be true, reasonable or
amount to knowledge. They focus on what is wrong with having some
combinations of beliefs, regardless of the substantive properties of
such beliefs.

\hypertarget{PREFIXaccuracy-dominance-and-consistency}{%
\section{Accuracy-Dominance and
Consistency}\label{PREFIXaccuracy-dominance-and-consistency}}

Accuracy-dominance arguments for vindicating the normativity of
Consistency come from decision theory and rely on the following
principle:

\textbf{Strong Dominance}. If an available state \(X\) is strongly
dominated by an available state \(Y\) at every possible world, in the
sense that state \(Y\) is better or has more value than state \(X\) at
every possible world, one ought to avoid state \(X\).

\noindent Strong Dominance has been used to vindicate probabilism, the
view roughly stating that an agent's rational credences should satisfy
the probability axioms. With respect to some inaccuracy measures such as
the Brier score, probabilistically inconsistent agents have access to a
credence function that is less inaccurate (and thus less epistemically
disvaluable) at every possible world {[}@joyce\_jm:1998;
@leitgeb-pettigrew:2010; @pettigrew:2016{]}.

For the sake of simplicity, I will leave aside dominance for credence
and focus on dominance for belief (these arguments have the same
structure, but dominance arguments for belief are more accessible).

There is a plausible explanation of why inconsistent combinations of
beliefs are strongly dominated. An agent can take different doxastic
attitudes towards \(P\), as in the following:

\begin{enumerate}
\def\labelenumi{(\roman{enumi})}
\item
  Believing \(P\) and not disbelieving \(P\),
\item
  Disbelieving \(P\) and not believing \(P\),
\item
  Neither believing nor disbelieving \(P\),
\item
  Believing \(P\) and disbelieving \(P\).
\end{enumerate}

The question is whether (iv) is strongly dominated. To answer this
question, we need to determine the epistemic value of (iv) at every
possible world. In veritistic frameworks, only true beliefs have final
epistemic value and only false beliefs have final epistemic disvalue.
Accordingly, \(T\) is the epistemic value of having a true belief (for
\(T > 0\)), F is the epistemic disvalue of having a false belief (for
\(F < 0\)), and the epistemic value of not believing \(P\) (or not
disbelieving \(P\)) is 0.\footnote{I'm glossing over some inessential
  subtleties here. It is possible to assign a value to not believing
  \(P\) (or to withholding judgment on whether P), but ultimately, we
  would get exactly the same results. See {[}@easwaran:2016b§C{]} and
  {[}@dorst:201910n12{]}.} Finally, assume that \(T \leftarrow F\),
which amounts to endorsing a conservative account of epistemic value.
The conservative constraint on epistemic value is plausible.\footnote{But
  this constraint might not stem from accuracy-first epistemology. See
  {[}@steinberger\_f:2019a{]} and the next footnote.} As Dorst
says:\footnote{In addition to Dorst's argument, see
  {[}@easwaran:2016b{]}, {[}@easwaran-fitelson:2015{]} and
  {[}@pettigrew:2016c{]} for similar arguments in favour of the
  conservative account of epistemic value. See
  {[}@steinberger\_f:2019a{]} on why alternatives to conservatism are
  compatible with accuracy-first epistemology.}

\begin{quote}
{[}An epistemically rational agent{]} will be doxastically
conservative\ldots{} Why? Well here's a fair coin --- does she believe
it'll land heads? Or tails? Or both? Or neither? Clearly neither. But if
she cared more about seeking truth than avoiding error, why not believe
both? She'd then be guaranteed to get one truth and one falsehood, and
so be more accurate than if she believed neither\ldots{} Upshot: we
impose a \emph{Conservativeness} constraint to capture the sense in
which Rachael has `more to lose' in forming a belief than she does to
gain. {[}@dorst:2019, 11{]}
\end{quote}

Then, we can determine the possible values of each option at every
possible world. Since the value of these options is solely determined by
P's truth value, we need to consider the worlds in which \(P\) is true
and the worlds in which \(P\) is false, as in the following table:

\begin{longtable}[]{@{}lll@{}}
\toprule
Table 1. & & \\
\midrule
\endhead
\textbf{Doxastic options / possible world} & \textbf{\(P\) is true} &
\textbf{\(P\) is false} \\
Believing \(P\) and not disbelieving \(P\) & \(T\) & \(F\) \\
Disbelieving \(P\) and not believing \(P\) & \(F\) & \(T\) \\
Neither believing nor disbelieving \(P\) & 0 & 0 \\
Believing \(P\) and disbelieving \(P\) & \(T+F\) & \(T+F\) \\
\bottomrule
\end{longtable}

Finally, in accordance with Table 1, we can conclude that inconsistent
combinations of beliefs are strongly dominated. The following reasoning
supports such a conclusion:

\begin{enumerate}
\def\labelenumi{(\arabic{enumi})}
\item
  \(T \leftarrow F\) (conservative assumption). Accordingly,
  \(T+F < 0\).
\item
  Following (1) and Table 1, believing \(P\) and disbelieving \(P\)
  simultaneously has an epistemic value of less than 0 at every possible
  world.
\item
  However, following Table 1, neither believing nor disbelieving \(P\)
  has an epistemic value of 0 at every possible world.
\end{enumerate}

\begin{enumerate}
\def\labelenumi{(\Alph{enumi})}
\setcounter{enumi}{2}
\tightlist
\item
  Therefore, following (2) and (3), inconsistent combinations of beliefs
  such as believing \(P\) and disbelieving \(P\) are strongly dominated:
  another available option (neither believing nor disbelieving \(P\)) is
  more valuable at every possible world.\footnote{Similar arguments can
    be found in {[}@easwaran:2016b§B{]} and {[}@pettigrew:2016c256{]}.
    {[}@dorst:201931 --- esp.~proposition 3{]} argues for a similar but
    contextualist view.}
\end{enumerate}

Hence, one ought to avoid being inconsistent.

\hypertarget{PREFIXtruth-conduciveness-reasons-for-and-reasons-against}{%
\section{Truth-Conduciveness, Reasons For and Reasons
Against}\label{PREFIXtruth-conduciveness-reasons-for-and-reasons-against}}

\hypertarget{PREFIXkolodnys-objection-from-truth-conduciveness}{%
\subsection{Kolodny's Objection From
Truth-Conduciveness}\label{PREFIXkolodnys-objection-from-truth-conduciveness}}

The above argument states that inconsistent combinations of beliefs are
dominated, which means that one ought not to be inconsistent. Naturally,
this seems to suggest that one ought to be consistent. But this
equivalence is less obvious than it seems.

To see why, consider Kolodny's argument against the normativity of
Consistency. According to him, one does not necessarily have an
epistemic reason to be consistent. Rather, what matters from an
epistemic point of view is having true beliefs and avoiding false
beliefs, and satisfying Consistency does not guarantee a better ratio of
true to false beliefs. In fact, some perfectly consistent sets of
beliefs are entirely false (or improbable on the evidence). Kolodny
summarizes his argument in the following way:

\begin{quote}
From the standpoint of theoretical deliberation --- which asks `What
ought I to believe?'---what ultimately matters is simply what is likely
to be true, given what there is to go on. {[}\ldots{]} {[}But{]} formal
coherence may as soon lead one away from, as toward, the true and the
good. Thus, if someone asks from the deliberative standpoint `What is
there to be said for making my attitudes formally coherent as such?'
there seems, on reflection, no satisfactory answer. {[}@kolodny:2007a,
231{]}
\end{quote}

In other words, if one merely satisfies Consistency, one is not more
likely to end up forming true beliefs and avoiding false beliefs. So,
the mere satisfaction of Consistency does not improve one's ratio of
true to false beliefs. In view of the foregoing, Kolodny thinks that it
is false that one falls under an obligation to be consistent.\footnote{Elsewhere,
  {[}@kolodny:2005{]} raises some objections against the normativity of
  other structural requirements, such as Inter-Level Coherence.}

\hypertarget{PREFIXcomparing-the-objection-from-truth-conduciveness-and-accuracy-dominance-arguments}{%
\subsection{Comparing the Objection from Truth-Conduciveness and
Accuracy-Dominance
Arguments}\label{PREFIXcomparing-the-objection-from-truth-conduciveness-and-accuracy-dominance-arguments}}

Kolodny argues that there is no reason to be consistent. His argument
relies on the fact that being consistent does not guarantee a good ratio
of true to false beliefs. By way of contrast, accuracy-dominance
arguments suggest that there is good reason not to be inconsistent. If
one is inconsistent, one is strongly dominated, in the sense that one
has access to a better option at every possible world. For instance, if
one believes \(P\) and disbelieves \(P\) simultaneously, one will
necessarily improve one's situation by neither believing nor
disbelieving P.

Accuracy-dominance arguments and Kolodny's objection from
truth-conduciveness are both veritistic.\footnote{See notably
  {[}@goldman\_ai:2015a{]} and {[}@whiting\_da:2010a{]} on veritism.}
Indeed, they presuppose that only true beliefs bear final epistemic
value, and only false beliefs bear final epistemic disvalue.
Nevertheless, such arguments apparently support incompatible conclusions
concerning the normativity of Consistency: Kolodny argues that veritism
entails the denial of the normativity of Consistency, whereas
accuracy-dominance arguments support the normativity of Consistency.
This is puzzling.

Perhaps Kolodny and accuracy-dominance theorists do not endorse the same
version of veritism. Veritism says that only true beliefs have final
epistemic value, and only false beliefs have final epistemic disvalue.
However, when it comes to epistemic obligations and permissions, these
assumptions concerning epistemic value might translate in many different
ways. For instance, perhaps agents ought to maximize their \emph{total}
epistemic score (e.g.~the total balance of epistemic value they get from
their doxastic states), or perhaps agents ought to maximize their
\emph{expected} epistemic score. For clarity, consider the following
example: Suppose \(P\) is very likely relative to a body of evidence E.
But as it happens, \(P\) is false. Then, believing \(P\) (or having a
high credence in P) might maximize expected epistemic value with respect
to E. But disbelieving \(P\) (or having a low credence in P) will
maximize epistemic value \emph{tout court}.

Yet, it is implausible that a difference in how Joyce and Kolodny
understand veritism is the reason why they disagree. Kolodny's argument
can be reformulated in many different ways. Consider the following
possibilities: (i) Suppose agents ought to maximize \emph{expected}
accuracy. Then, Kolodny could say: Some consistent combinations of
beliefs can minimize expected accuracy (believing the most improbable
propositions can be consistent). (ii) Suppose agents ought to optimize
their ratio of true to false beliefs. Then, Kolodny could argue that
some agents with a very bad ratio of true to false beliefs are
consistent. (iii) Suppose agents ought to maximize \emph{total}
accuracy. Then, Kolodny could say: Some consistent combinations of
beliefs can minimize accuracy (believing false propositions only can be
consistent). As we can see, Kolodny's objection is malleable.\footnote{I
  thank a reviewer for inviting me to discuss this possibility.}

Another possibility is that Kolodny and accuracy-first theorists have a
different understanding of what ``ought'' means. We can make a
distinction between normativity in the rule-following sense (as in:
Relative to domain D, A ought to X) and normativity in the
reason-involving sense (as in: A has a reason to X).\footnote{See
  {[}@parfit:2011144--48{]} on this distinction.} For example, the rules
of etiquette require of agents to be polite, but agents might lack a
reason to be polite. By way of analogy with the rules of etiquette,
perhaps accuracy-first theorists are merely interested in arguing that
the rules of rationality require consistency. This would be compatible
with Kolodny's view --- namely, that agents do not have a reason to be
consistent. Both views would then be compatible with each other.

It is true that accuracy-first theorists see Consistency as a demand of
rationality. However, it is implausible that accuracy-first theorists
are \emph{merely} concerned with normativity in the rule-following
sense. Accuracy-first theorists like Joyce tie norms of rationality to
epistemic value, as in the following:

\textbf{The Norm of Truth}. An epistemically rational agent must strive
to hold a system of full beliefs that strikes the best attainable
overall balance between the epistemic good of fully believing truths and
the epistemic evil of fully believing falsehoods {[}@joyce\_jm:1998,
577{]}.

\textbf{The Norm of Gradational Accuracy.} An epistemically rational
agent must evaluate partial beliefs on the basis of their gradational
accuracy, and she must strive to hold a system of partial beliefs that,
in her best judgment, is likely to have an overall level of gradational
accuracy at least as high as that of any alternative system she might
adopt {[}@joyce\_jm:1998, 579{]}.

\noindent Satisfying the requirements of rationality is different from,
say, satisfying the requirements of etiquette. The former has a
privileged relationship to value. Epistemically rational agents want to
optimize their overall balance of epistemic value. Accordingly, it would
be surprising that Joyce and others are merely concerned with
normativity in the rule-following sense. Specifically, it would be
surprising that, while rationality has some sort of privileged
relationship to value, it is merely normative in the rule-following
sense.\footnote{I thank a reviewer for inviting me to clarify this
  possibility.}

Under the assumption that Kolodny and accuracy-dominance theorists agree
upon a specific version of veritism and the meaning of ``ought,'' the
natural reaction is to think that at least one of the above arguments is
mistaken --- either the objection from truth-conduciveness is
inconclusive, or accuracy-dominance arguments fail. After all, how can
there be no reason to be consistent and reasons against being
inconsistent? If there is something wrong with being inconsistent, there
must be something good with being consistent!

However, this natural reaction presupposes that there is always a
connection between (i) reasons for being consistent (as in Normativity+)
and (ii) reasons against being inconsistent (as in Normativity-). Call
this the Coextensivity Thesis, as in the following:

\textbf{Coextensivity Thesis}. Arguments in favour of Normativity- count
as arguments in favour of Normativity+ (and vice versa).

\noindent Those who endorse the Coextensivity Thesis think that (i) and
(ii) express the same normative relation.

If the Coextensivity Thesis were correct, then Kolodny's objection from
truth-conduciveness would be inconclusive. Under the assumption that the
Coextensivity Thesis is correct, two kinds of considerations can
vindicate the view that one ought to be consistent --- namely, reasons
be consistent and reasons against being inconsistent. Kolodny argues for
the \emph{absence} of reasons in favour of being consistent. But if the
Coextensivity Thesis is correct, \emph{such considerations are just half
of the story}. We also need to consider whether there are reasons
against being inconsistent in the balance, since they count as reasons
for being consistent. Accuracy-dominance arguments entail that one ought
not to be inconsistent. So, even if Kolodny is right that there is no
reason to satisfy Consistency, this does not entail that it is false
that one ought to be consistent. Insofar as there are arguments against
inconsistency (as suggested by accuracy-dominance arguments), there is a
reason to be consistent.

However, if the Coextensivity Thesis is false, then accuracy-dominance
arguments are compatible with the objection from truth-conduciveness.
Here is why. Kolodny argues that there is no reason to be consistent: he
denies that one ought to be consistent, as in Normativity+. However, if
the Coextensivity Thesis is false, we can deny Normativity+ without
denying Normativity-. In other words, even if it is false that one ought
to be consistent, perhaps one ought not to be inconsistent. The same
goes for accuracy-dominance arguments. According to such arguments,
inconsistent combinations of beliefs are dominated. So, one ought not to
be inconsistent. But if the Coextensivity Thesis is false, this does not
entail that one ought to be consistent.

\hypertarget{PREFIXreasons-to-be-consistent-and-the-coextensivity-thesis}{%
\subsection{Reasons to be Consistent and the Coextensivity
Thesis}\label{PREFIXreasons-to-be-consistent-and-the-coextensivity-thesis}}

So, is the Coextensivity Thesis true? This depends on what ``a reason to
be consistent'' means. Suppose, like Kolodny, that ``a reason to be
consistent'' concerns each individual consistent option one has (see
§3.1). That is, suppose that ``a reason to be consistent'' means
something like ``a consideration that counts in favour of \emph{each}
individual consistent options one has.'' For Kolodny, nothing can be
said in favour of some consistent combinations of attitudes. So, under
this interpretation of what ``a reason to be consistent'' means, we do
not necessarily have a reason to be consistent.

Relative to this interpretation of what ``a reason to be consistent''
means, the Coextensitivity Thesis does not seem plausible. For reasons
found in {[}@snedegar\_j:2018{]}, we can make a distinction between
reasons for Consistency (as in Normativity+) and reasons against
inconsistency (as in Normativity-). The distinction comes from the
following account of reasons for and reasons against endorsed by
Snedegar:

\begin{quote}
My view puts a strong condition on reasons for and a weak condition on
reasons against. For some objective to provide a reason for an option,
that option has to do the best with respect to the objective. For some
objective to provide a reason against an option, that option only has to
do worse than some alternative. {[}@snedegar\_j:2018, 737{]}
\end{quote}

Snedegar roughly argues that the problem with views that lump together
reasons against and reasons for is that there can be good reasons not to
\(\phi\), even if there are worse alternatives to
\(\phi\)-ing.\footnote{See {[}@snedegar\_j:2018{]} for more details.}
For instance, suppose that I am trying to decide what to drink. I might
have conclusive reason not to drink gin, but this does not entail that I
have a reason to drink any beverage that isn't gin. I should definitely
not drink petrol, even if petrol isn't gin. This is compatible with my
having conclusive reason not to drink gin.

Snedegar's observation sits well with accuracy-dominance arguments
discussed in Sec.~\ref{sec:PREFIXaccuracy-dominance-and-consistency}.
Indeed, recall the options agents have in table 1:

\begin{longtable}[]{@{}lll@{}}
\toprule
Table 1. & & \\
\midrule
\endhead
\textbf{Doxastic options / possible world} & \textbf{P is true} &
\textbf{P is false} \\
Believing \(P\) and not disbelieving \(P\) & T & F \\
Disbelieving \(P\) and not believing \(P\) & F & T \\
Neither believing nor disbelieving \(P\) & 0 & 0 \\
Believing \(P\) and disbelieving \(P\) & T+F & T+F \\
\bottomrule
\end{longtable}

\noindent Clearly, there is conclusive reason not to go for the
inconsistent option, since neither believing nor disbelieving \(P\) is
better than being inconsistent at every possible world. However, this
does not entail that there is a reason in favour of every alternative to
the inconsistent option. For instance, disbelieving \(P\) when \(P\) is
true (or believing \(P\) when P is false) is worse than being
inconsistent. So, as in the gin and petrol case, reasons against
inconsistency are logically weaker than reasons for Consistency.

This suggests that accuracy-dominance arguments do not vindicate
Normativity+ on their own. Of course, when combined with the
Coextensivity Thesis, these arguments support Normativity+. But
Kolodny's interpretation of what ``a reason to be consistent'' means
conflicts with the Coextensivity Thesis. So, while accuracy-dominance
arguments support Normativity-, it is an open question whether they also
support Normativity+.

Here is a response to my argument on behalf of the accuracy-dominance
theorist. We can regroup the consistent options in table 1 under a
single option. Call this the consistent option. With respect to the
consistent option, Snedegar's distinction does not apply. If there is
conclusive reason not to go for the inconsistent option, and the only
option left is the ``regrouped'' consistent option, then reasons against
inconsistency favour the consistent option. So, could there be a sense
in which the Coextensivity Thesis is true?\footnote{I thank a referee
  for inviting me to discuss this objection.}

My response to this objection goes as follows. This way of framing the
problem cannot make sense of Kolodny's objection concerning some
consistent options. \emph{There is something wrong with some consistent
combinations of beliefs} --- some consistent combinations of beliefs are
entirely wrong or improbable on the evidence. Kolodny is right to point
out that nothing can be said in favour of these combinations of
attitudes. The only way to make sense of Kolodny's objection is
\emph{not} to regroup all the consistent options under a single label,
precisely because relevant normative distinctions can (and should) be
made between some consistent options.

At best, this reply shows that, under a different interpretation of what
``a reason to be consistent'' means, the Coextensivity Thesis is true.
But Kolodny's argument still succeeds relative to another interpretation
of this expression. When Kolodny discusses the normativity of
Consistency, he discusses the normativity of the individual consistent
options one has, including the ones that are entirely wrong or
improbable on the evidence. The accuracy-dominance theorist can claim
that one ought to be consistent, but that is simply because the
expression ``one ought to be consistent'' here refers to something
logically weaker than what Kolodny has in mind.\footnote{My response
  might not convince some readers. In any case, we can draw a lesson
  from this discussion. We have learned that the expression ``a reason
  to be consistent'' is ambiguous. Some readings of this expression are
  a problem for Kolodny's argument, and other readings of this
  expression conflict with vindicating Normativity+.}

\hypertarget{PREFIXan-escape-route-for-the-accuracy-dominance-theorist}{%
\subsection{An Escape Route for the Accuracy-Dominance
Theorist?}\label{PREFIXan-escape-route-for-the-accuracy-dominance-theorist}}

The accuracy-dominance theorist could then offer the following
objection. Suppose there is an accuracy-dominance argument against one's
attitudes. Accordingly, one can identify at least one collection of
attitudes that veritistically dominates one's current state. If agents
can identify at least one set of attitudes that is better than their
current state, then they have a reason to take the dominating set of
attitudes, which will be consistent. Doesn't this support the view
according to which one ought to be consistent? If agents ought to take
dominating combinations of beliefs, and such combinations of beliefs are
consistent, then this seems to entail that agents ought to be
consistent.\footnote{I thank a reviewer for bringing this objection to
  my attention.}

This objection carries weight depending on what accuracy-dominance
arguments prove. Let me explain.

Suppose the contender is right. Then, accuracy-dominance vindications
are akin to the Truth Vindication, the Knowledge Vindication or the
Reasons Vindication discussed in
sec.~\ref{sec:PREFIXthe-why-be-consistent-challenges}. If one has
inconsistent combinations of beliefs (say, one believes \(P\) and also
believes \(\neg\)P), the Truth Vindication says that agents ought to
maintain the true one (and abandon the false one), the Knowledge
Vindication says that agents are only permitted to maintain the known
one, and the Reasons Vindication says that agents are only permitted to
maintain the reasonable one (and ought to abandon the
\emph{unreasonable} one). In any case, satisfying such norms means that
agents will cease entertaining inconsistent combinations of beliefs.

The contender makes a similar point. If one has inconsistent
combinations of beliefs, one should go for the option dominating
inconsistent combinations of beliefs. But if that is right, the
accuracy-dominance argument merely entails that agents ought (or have
reasons) to have \emph{some} combinations of beliefs, not \emph{any}
consistent combination of beliefs. In other words, the argument leaves
out some consistent combinations of beliefs.

This brings us back to the discussion in
Sec.~\ref{sec:PREFIXthe-why-be-consistent-challenges}. What do we expect
from a good vindication of the normativity of Consistency? For many
philosophers, a good vindication of Consistency should cover all the
possible consistent combinations of beliefs. If the contender is right,
then accuracy-dominance arguments can explain the significance of some
consistent combinations of beliefs --- namely, the dominating ones. But
this is not what we were looking for. The explanation should apply to
\emph{all} the consistent combination of beliefs. To be clear: Some
philosophers might not be interested in this specific interpretation of
the ``Why-Be-Consistent'' debate. It should be clear that, with respect
to other understandings of the question, the contender is right.

\hypertarget{PREFIXconclusion-and-implications-in-the-debate-on-the-normativity-of-structural-rationality}{%
\section{Conclusion and Implications in the Debate On The Normativity of
Structural
Rationality}\label{PREFIXconclusion-and-implications-in-the-debate-on-the-normativity-of-structural-rationality}}

This paper supports the view that there are two theses concerning the
normativity of Consistency: Normativity+ and Normativity-. While
accuracy-dominance arguments support Normativity-, they might not
necessarily support Normativity+. This is so, because the Coextensivity
Thesis might be false. In fact, one way to reconcile Kolodny's objection
from truth-conduciveness with accuracy-dominance arguments is to deny
the Coextensivity Thesis.

These clarifications concerning Normativity+ and Normativity- allow us
to rethink the debate on the normativity of structural rationality.
Indeed, a popular strategy for arguing against the normativity of
structural rationality is to point out that there is no reason to
satisfy some specific rational requirements (such as Consistency).
Kolodny's objection from truth-conduciveness is a good illustration of
such arguments. These arguments are compelling if we focus on
Normativity+. But this might be a mistake. Perhaps that, when it comes
to formal requirements like Consistency, the only view we should try to
vindicate is Normativity-.

The argument of this paper allows us to make sense of some
pre-theoretically correct assumptions structural requirements of
epistemic rationality such as Consistency. Plausibly, there is something
wrong, suboptimal or disvaluable with inconsistent combinations of
beliefs. The mistake might have been to try to explain this assumption
in terms of \emph{an obligation to be consistent}. But if I am right, we
might only be able to explain this assumption in terms of \emph{an
obligation not to be inconsistent}. Hence, requirements like Consistency
might merely be normative in a weak sense.

The good news is that we can now make sense of such a possibility. If
the Coextensivity Thesis is false, it makes perfect sense to say that
one ought not to be inconsistent without also saying that one ought to
be consistent. There might not be something good with being structurally
rational, but it seems patently clear that there is something bad with
being structurally irrational.

\hypertarget{PREFIXreferences}{%
\section{References}\label{PREFIXreferences}}

\end{document}
