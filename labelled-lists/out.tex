Labelled-lists =======

custom labelled lists in Pandoc's markdown.

v0.1. Copyright: © 2021 Julien Dutant
\href{mailto:julien.dutant@kcl.ac.uk}{\nolinkurl{julien.dutant@kcl.ac.uk}}
License: MIT - see LICENSE file for details.

\hypertarget{introduction}{%
\subsection{Introduction}\label{introduction}}

This filter provides custom labelled lists in Pandoc's markdown for
outputs in LaTeX/PDF, HTML and JATS XML. Instead of bullets or numbers,
list items are given custom text labels. The text labels can include
markdown formatting.

\hypertarget{usage}{%
\subsection{Usage}\label{usage}}

\hypertarget{loading-the-filter}{%
\subsubsection{Loading the filter}\label{loading-the-filter}}

The filter is loaded with the Pandoc \texttt{-L} or
\texttt{-\/-lua-filter} option.

\texttt{bash\ pandoc\ -L\ path/to/labelled-lists.lua\ source.md\ -o\ output.html}

If the filter is in Pandoc's \texttt{\$DATADIR} there is no need to give
its path. See {[}Pandoc's manual{[}
(https://pandoc.org/MANUAL.html\#general-options) for details.

\hypertarget{markdown-syntax}{%
\subsubsection{Markdown syntax}\label{markdown-syntax}}

A simple illustration of the custom label syntax:

\begin{Shaded}
\begin{Highlighting}[]
\SpecialStringTok{* }\CommentTok{[}\OtherTok{Premise 1}\CommentTok{]}\NormalTok{\{\} This is the first claim.}
\SpecialStringTok{* }\CommentTok{[}\OtherTok{Premise 2}\CommentTok{]}\NormalTok{\{\} This is the second claim.}
\SpecialStringTok{* }\CommentTok{[}\OtherTok{Conclusion}\CommentTok{]}\NormalTok{\{\} This is the conclusion.}
\end{Highlighting}
\end{Shaded}

This generates the following list (process this file with the filter to
see the result):

\begin{itemize}
\tightlist

\item[(Premise 1)]

This is the first claim.

\item[(Premise 2)]

This is the second claim.

\item[(Conclusion)]

This is the conclusion.

\end{itemize}

In general, the filter will turn a bullet list into a custom label list
provided that \emph{every item starts with a non-empty Span element}.

\begin{itemize}
\tightlist
\item
  A Span element is inline text (i.e., not block like a paragraph) that
  optinally has some attributes. The default syntax is
  \texttt{{[}inline\ text{]}\{attributes\}}. Inline text will be used as
  label, placed within round bracket.
\item
  There is no need to specify attributes on the Span. But (in Pandoc's
  default Span syntax) curly brackets must be present:
  \texttt{{[}label{]}} won't work, \texttt{{[}label{]}\{\}} will.
\item
  For the purposes of this filter, a Span is \emph{non-empty} if its
  inline text is not empty. Thus \texttt{{[}{]}\{\}} will not work.
  Numbers or other unicode characters work. To generate an empty label,
  use a space or other invisible character,
  e.g.~\texttt{{[}\ {]}\ \ \ \{\}}. \emph{Exception}: math formulas will
  work as labels, but if the inline text only contains LaTeX code
  (\texttt{\textbackslash{}textsc\{a\}}) it will be treated as empty.
\item
  The label can include formatting. \texttt{{[}**T1**{]}\{\}} will
  generate a label with strong emphasis (bold by default).
\item
  Alternative syntaxes for Span elements will work too. See {[}Pandoc
  manual{]} (https://pandoc.org/MANUAL.html\#divs-and-spans) for
  details.
\end{itemize}

\hypertarget{examples-and-tests}{%
\subsection{Examples and tests}\label{examples-and-tests}}

\hypertarget{math-formulas}{%
\subsubsection{math formulas}\label{math-formulas}}

\begin{itemize}
\tightlist

\item[(\(p_1\))]

This list uses

\item[(\(p_2\))]

math formulas as labels.

\end{itemize}

\hypertarget{latex-code}{%
\subsubsection{LaTeX code}\label{latex-code}}

\begin{itemize}
\tightlist
\item
  {\textbf{a}} This list uses
\item
  {\textbf{b}} latex code as labels.
\end{itemize}

Ignored: these are not treated as labels.

\hypertarget{small-caps}{%
\subsubsection{Small caps}\label{small-caps}}

\begin{itemize}
\tightlist

\item[(\textsc{All})]

This list uses

\item[(\textsc{Some})]

latex code as labels.

\end{itemize}

\hypertarget{list-with-para-items}{%
\subsubsection{List with Para items}\label{list-with-para-items}}

\begin{itemize}
\tightlist

\item[(A1)]

\[F(x) > G(x)\]

\item[(A2)]

\[G(x) > H(x)\]

\end{itemize}

\hypertarget{items-with-several-blocks}{%
\subsubsection{items with several
blocks}\label{items-with-several-blocks}}

\begin{itemize}
\tightlist

\item[(\textbf{B1})]

This list's items

consist of several blocks

\[\sum_i Fi > \sum_i Gi\]

\item[(\textbf{B2})]

Lorem ipsum dolor sit amet, consectetur adipiscing elit. Donec et massa
ut eros volutpat gravida ut vel lacus. Proin turpis eros, imperdiet sed
quam eget, bibendum aliquam massa. Phasellus pellentesque egestas
dapibus. Proin porta tellus id orci consectetur bibendum. Nam eu cursus
quam. Etiam vehicula in mi sed interdum. Duis rutrum eleifend
consectetur. Phasellus ullamcorper, urna at vestibulum venenatis, tellus
erat luctus nibh, eget hendrerit justo enim nec magna. Duis mollis ac
felis ac tristique.

Pellentesque malesuada arcu ac orci scelerisque vulputate. Aenean at ex
suscipit, ultricies tellus sit amet, luctus lectus. Duis ut viverra
sapien. Pellentesque habitant morbi tristique senectus et netus et
malesuada fames ac turpis egestas. Cras consequat nisi at ex finibus, in
condimentum erat auctor. In at nulla at est iaculis pulvinar sed id
diam. Cras malesuada sit amet tellus id molestie.

\end{itemize}

\hypertarget{details}{%
\subsection{Details}\label{details}}

\hypertarget{latex-output}{%
\subsubsection{LaTeX output}\label{latex-output}}

\begin{Shaded}
\begin{Highlighting}[]
\KeywordTok{\textbackslash{}begin}\NormalTok{\{}\ExtensionTok{itemize}\NormalTok{\}}
\FunctionTok{\textbackslash{}tightlist}

\FunctionTok{\textbackslash{}item}\NormalTok{[(Premise 1)]}

\NormalTok{This is the first claim.}

\FunctionTok{\textbackslash{}item}\NormalTok{[(Premise 2)]}

\NormalTok{This is the second claim.}

\FunctionTok{\textbackslash{}item}\NormalTok{[(Conclusion)]}

\NormalTok{This is the conclusion.}

\KeywordTok{\textbackslash{}end}\NormalTok{\{}\ExtensionTok{itemize}\NormalTok{\}}
\end{Highlighting}
\end{Shaded}

\hypertarget{html-output}{%
\subsubsection{HTML output}\label{html-output}}

\begin{Shaded}
\begin{Highlighting}[]
\KeywordTok{\textless{}div} \ErrorTok{class}\OtherTok{=}\StringTok{"labelled{-}lists{-}list"}\KeywordTok{\textgreater{}}
    \KeywordTok{\textless{}div} \ErrorTok{class}\OtherTok{=}\StringTok{"labelled{-}lists{-}item"}\KeywordTok{\textgreater{}}
      \KeywordTok{\textless{}div\textgreater{}\textless{}span} \ErrorTok{class}\OtherTok{=}\StringTok{"labelled{-}lists{-}label"}\KeywordTok{\textgreater{}}\NormalTok{(Premise 1)}\KeywordTok{\textless{}/span\textgreater{}}\NormalTok{ Text }
\NormalTok{      of the item\textquotesingle{}s first block}\ErrorTok{\textless{}}\NormalTok{\textbackslash{}div\textgreater{}}
      \KeywordTok{\textless{}p\textgreater{}}\NormalTok{text of the item\textquotesingle{}s second paragraph}\KeywordTok{\textless{}/p\textgreater{}}
    \KeywordTok{\textless{}div\textgreater{}} 
    \KeywordTok{\textless{}div} \ErrorTok{class}\OtherTok{=}\StringTok{"labelled{-}lists{-}item"}\KeywordTok{\textgreater{}}
      \KeywordTok{\textless{}div\textgreater{}\textless{}span} \ErrorTok{class}\OtherTok{=}\StringTok{"labelled{-}lists{-}label"}\KeywordTok{\textgreater{}}\NormalTok{(Premise 2)}\KeywordTok{\textless{}/span\textgreater{}}\NormalTok{ Text }
\NormalTok{      of the item\textquotesingle{}s first block}\ErrorTok{\textless{}}\NormalTok{\textbackslash{}div\textgreater{}}
      \KeywordTok{\textless{}p\textgreater{}}\NormalTok{text of the item\textquotesingle{}s second paragraph}\KeywordTok{\textless{}/p\textgreater{}}
    \KeywordTok{\textless{}div\textgreater{}} 
\end{Highlighting}
\end{Shaded}

\hypertarget{list-structures}{%
\subsubsection{List structures}\label{list-structures}}

\begin{itemize}
\tightlist
\item
  In the Pandoc AST, each item is a list of blocks. If the item has only
  one block, the list will contain only one element.
\item
  If an item has only one block, that block's type can be at least:

  \begin{itemize}
  \tightlist
  \item
    Plain, if it only contains straightforward markdown
  \item
    Para, if it contains some equation LaTeX code (and perhaps in other
    cases too)
  \item
    Table if it contains a table.
  \end{itemize}
\item
  It an item has several blocks, they will be Para by default, otherwise
  of whatever type the block is.
\item
  If an item is only one block, it is either a Plain element (if it only
  contains straightforward markdown) or a Para element (if it contains
  some LaTeX code or equation)
\end{itemize}
